\documentclass{article}

\usepackage{graphicx}
\usepackage{tikz,pgfplots}
\usepackage{preview}	
\usepackage{mathtools}
\usepackage{amsmath}
\usepackage{amssymb}
\usepackage{amsthm}
\usepackage[english]{babel}
\usepackage[utf8]{inputenc}
\usepackage[english]{babel}	
\usepackage{natbib}
\usepackage{color}
\usepackage[a4paper,top=3cm,bottom=3cm, right=2cm, left=2cm]{geometry}
\usepackage[normalem]{ulem}
\usepackage{blindtext}
\usepackage{xcolor}
\usepackage[colorinlistoftodos]{todonotes}
\usepackage[colorlinks=true, allcolors=blue]{hyperref}
\usepackage{cleveref} %label the theorem

\usetikzlibrary{math}

\bibliographystyle{agsm}

\newtheorem{theorem}{Theorem}	
\newtheorem{corollary}{Corollary}
\newtheorem{proposition}{Proposition}
\newtheorem{observation}{Observation}
\newtheorem{assumption}{Assumption}	
\newtheorem{definition}{Definition}
\newtheorem{remark}{Remark}
\newtheorem{lemma}{Lemma}
\newtheorem{result}{result}

\DeclarePairedDelimiter\ceil{\lceil}{\rceil}
\DeclarePairedDelimiter\floor{\lfloor}{\rfloor}

\date{October 2018}

\begin{document}

\textcolor{red}{When will someone prefer the future payoffs over the present ones? }

\textcolor{green}{Intuitively, If the mean, $\mu<0$ is negative and increasing over time $\mu'(t)>0$, it seems intuitive that one should prefer future payouts }

\textcolor{red}{Is there an elegant way to show this formally? }

%For the early payout to be prefer-ed  to the future payout we must have:

%\begin{align*}
%x(t)e^{\mu \Delta T}+ x_a e^{\mu \Delta (T-t_a)} > x(t)e^{\mu \Delta T}+ x_b e^{\mu \Delta (T-t_b)} \\ 
%\rightarrow \Delta x_a^{\mu (\Delta T- t_a)}
%-\Delta x_b^{\mu (\Delta T- t_b)}>0
%\end{align*}

\section{Cases and growth rates}

The goal here is to simply give a taxonomy of evaluating gambles

\subsection{Repeat after $t_b$ or after some fixed period $t_o$ without re-investment }

\begin{align*}
g_a&=\frac{1}{\Delta t_b} ln 
\left( 
\frac{x(t)e^{(\mu \Delta t_b)}+\Delta x_a }{x(t)}
\right) \\
g_b&=\frac{1}{\Delta t_b} ln 
\left( 
\frac{x(t)e^{(\mu \Delta t_b)}+\Delta x_b }{x(t)}
\right) 
\end{align*}

A more general way to interpret this scheme is that somebody has fixed periods where they can invest and these periods do not depend on what bets the person currently has going on. Suppose the bets can be taken every $t_o$ then we prefer the first bet if: 

\begin{align*}
\frac{\Delta x_a}{t_o}&>\frac{\Delta x_b}{t_o} \\
\Delta x_a&>\Delta x_b 
\end{align*}

Which is never verified. So if we can only undertake bets at a specific period in time, we only compare the expected value. 

\textcolor{red}{What is the implied dynamic here? Clearly if $\mu=0$ then the implied dynamic is $dx = 0$}

\subsection{Repeat after $t_b$ or after some fixed period $t_o$ with re-investment }

If investment occur every fixed period in time, say a month, and this period of investment is independent of what bets are ongoing then we are in a very specific type of world. For instance, if investments come up very often, before a whole other investments are realized, then this implies we are not liquidity constrained, we can continuously undertake numerous investments at the same time. Additionally when the investments are realized they are re-invested into some rate which is identical to the rate which the rest of the wealth follows. 

\begin{align*}
g_a&= \frac{1}{\Delta t_b} ln 
\left( 
\frac{x(t)e^{(\mu \Delta t_b)}+\Delta x_a e^{\mu (\Delta t_b - \Delta t_a)}}{x(t)}
\right) \\
g_b&=\frac{1}{\Delta t_b} ln 
\left( 
\frac{x(t)e^{(\mu \Delta t_b)}+\Delta x_b }{x(t)}
\right) 
\end{align*}

So to compare with them we have to compare: 

\begin{align*}
g_a&>g_b \\
\rightarrow \Delta x_a e^{\mu (\Delta t_b-\Delta t_a)}&>\Delta x_b
\end{align*}

Note that this is the classic case

\subsection{Both bets repeat after they are realized without re-investment}

In this case the intuition is that only one bet can be undertaken at a time. As soon as the bet is realized then the bet is undertaken again. So this explains why if the bet is to be realized instantly, then we get infinite wealth! 

\begin{align*}
\frac{\Delta x_a}{\Delta t_a} > \frac{\Delta x_b}{\Delta t_b}
\end{align*}

\subsubsection{Preference reversal as a function of time}

\begin{align*}
H(\Delta x_a; t_{a}) = \frac{\Delta x_a }{t_a} \\
H(\Delta x_b;t_{b}) = 
\frac{\Delta x_b}{t_b} 
\end{align*}

Suppose $\Delta x_b>\Delta x_a$ and $t_{b}>t_a$ I think these two are sufficient for preference reversal. If $\tau$ time has elapsed then:

\begin{align*}
H(\Delta x_a; t_0, t_{a}-\tau) &= \frac{\Delta x_a}{t_a-\tau} \\
H(\Delta x_b;t_{b}-\tau) &= \frac{\Delta x_b}{t_b-\tau} 
\end{align*}

Therefore the discount rate is entirely described by the time horizon. 

Similarly the $\tau$ for which we are indifferent between the two options is given by:

\begin{align*}
\frac{\Delta x_a}{t_a-\tau}&=\frac{\Delta x_b}{t_b-\tau} \\
\Delta x_a(t_b-\tau)&=\Delta x_b(t_a-\tau) \\
\rightarrow \tau(\Delta x_a-\Delta x_b) &= \Delta x_bt_a- \Delta x_at_b \\
\rightarrow \tau &= \frac{\Delta x_bt_a- \Delta x_at_b}{(\Delta x_b-\Delta x_a)} 
\end{align*}

\subsection{Investments repeat after they are realized with re-investment}

\textcolor{red}{What is the criteria in this case? }

After $\Delta t_a $ has elapsed, the growth rate would have been: 

\begin{align*}
g_a&= \frac{1}{\Delta t_a} ln 
\left( 
\frac{x(t)e^{(\mu \Delta t_a)}+\Delta x_a }{x(t)}
\right) \\
g_b&=\frac{1}{\Delta t_a} ln 
\left( 
\frac{x(t)e^{(\mu \Delta t_a)} }{x(t)}
\right) 
\end{align*}

\textcolor{red}{ This case is harder to flesh out because we need to know the ratio $\frac{\Delta t_b}{ \Delta t_a}$, this is intuitive, for example if have that $2t_a=t_b$ we can compare the following:} 

\begin{align*}
g_a&= \frac{1}{2 \Delta t_a} ln 
\left( 
\frac{x(t)e^{(\mu 2\Delta t_a)}+\Delta x_ae^{\mu (2\Delta t_a - \Delta t_a)}+ \Delta x_a}{x(t)}
\right)
%%%%%%%%%%%%%%%%%%%%%%%%%%%%%
= \frac{1}{2 \Delta t_a} ln 
\left( 
\frac{x(t)e^{(\mu 2 \Delta t_a)}+\Delta x_ae^{\mu \Delta t_a }+ \Delta x_a}{x(t)}
\right) \\
%%%%%%%%%%%%%%%%%%%%%%%%%%%%%
g_b&=\frac{1}{\Delta t_b} ln 
\left( 
\frac{x(t)e^{(\mu \Delta t_b)} +\Delta x_b}{x(t)}
\right) 
\end{align*}

\textcolor{red}{ If have that $3t_a=t_b$ we can compare the following:} 

\begin{align*}
g_a&= \frac{1}{3\Delta t_a} ln 
\left( 
\frac{x(t)e^{(\mu 3 \Delta t_a)}+\Delta x_ae^{\mu 2\Delta t_a }+ \Delta x_a e^{\mu \Delta t_a }+\Delta x_a}{x(t)}
\right)   \\
%%%%%%%%%%%%%%%%%%%%%%%%%%%%%
g_b&=\frac{1}{\Delta t_b} ln 
\left( 
\frac{x(t)e^{(\mu \Delta t_b)} +\Delta x_b}{x(t)}
\right) 
\end{align*}

\textcolor{red}{ To generalize if  $nt_a=t_b$ we can compare the following, is there a more elegant generalization? Note that there could also be issues with the floor function $\floor*{\frac{\Delta x_a}{\Delta x_a}}$ because if the multiple is not exact there could be a ceiling function $\ceil*{\frac{\Delta x_a}{\Delta x_a}}$ } 


\begin{align*}
g_a&= \frac{1}{n\Delta t_a} ln 
\left( 
\frac{x(t)e^{(\mu n\Delta t_a)}+\Sigma ^n_{i=1}\Delta x_a e^{\mu (i \Delta t_a - \Delta t_a)}}{x(t)}
\right)   \\
%%%%%%%%%%%%%%%%%%%%%%%%%%%%%
g_b&=\frac{1}{\Delta t_b} ln 
\left( 
\frac{x(t)e^{(\mu \Delta t_b)} +\Delta x_b}{x(t)}
\right) 
\end{align*}

So to choose $\Delta x_a$ we must have that:

\begin{equation*}
\Sigma ^n_{i=1}\Delta x_a e^{\mu (i \Delta t_a - \Delta t_a)}>x(t)
\end{equation*}

\subsection{Summary table}

\begin{center}
\begin{tabular}{ |c|c|c| } 
 \hline
  & Repeated after $t_b$/not repeated & Repeated after they are realized \\ 
 re-invested & classic
 & \textcolor{red}{Unclear} \\ 
 not-re-invested & Expected Value of gamble, independent of time & Rate over time \\ 
 \hline
\end{tabular}
\end{center}

If the bets are repeating after they are realized this can be interpreted as more realistic because there will never be an issue of bankruptcy. That is, suppose that undertaking the investment has a fixed cost that is affordable. If the gamble repeats upon realization of the gamble then if the first gamble is affordable so are all future gambles. On the other hand if bets are repeating periodically, independently of what bets are currently being undertaken then it may not be realistic to evaluate in this way because the gambler will be tying too much liquidity in each bet. 

\end{document}
